% By zmienic jezyk na angielski/polski, dodaj opcje do klasy english lub polish
\documentclass[polish,12pt]{aghthesis}
\usepackage[utf8]{inputenc}
\usepackage{url}


\author{Adam Chrabąszcz, Konrad Kras}

\title{Biblioteka do lokalizacji robotów mobilnych ligi FIRA MiroSot}

\supervisor{dr inż. Wojciech Turek}

\date{2012}

% Szablon przystosowany jest do druku dwustronnego, 
\begin{document}

\maketitle



\section{Cel prac i wizja produktu}
%\section{Project goals and vision}
\label{sec:cel-wizja}
% \emph{Charakterystyka problemu, motywacja projektu (w tym przegląd
%   istniejących rozwiązań prowadząca do uzasadnienia celu prac), ogólna
%   wizja produktu, krótkie studium wykonalności i analiza zagrożeń.}
Celem projektu jest
dostarczenie systemu umożliwiającego lokalizację w czasie rzeczywistym
pozycji robotów oraz piłki na obrazie z kamery nad boiskiem gry zgodnie z
regułami ligi FIRA MiroSot Middle League.

Liga FIRA MiroSot Middle League określa zbiór zasad gry, w której drużyny
składające się z pięciu 7.5cm robotów poruszających się po prostokątnym boisku
220cm na 180cm próbują umieścić piłkę golfową w bramce przeciwnika. Roboty danej
drużyny sterowane są za pomocą jednego komputera, który uzyskuje informacje
o sytuacji na boisku przy pomocy kamery znajdującej się centralnie nad boiskiem.

W Laboratorium Robotów Katedry Informatyki znajdują się wszelkie fizyczne zasoby
niezbędne do rozegrania meczu zgodnie z zasadami ligi: zestaw robotów, 
odpowiednie boisko oraz kamera. Rozgrywanie meczy uniemożliwia brak
odpowiedniego oprogramowania sterującego robotami. W skład takiego
oprogramowania musi wchodzić system rozpoznający obraz z kamery.

Aktualnie w laboratorium działa system lokalizacji robotów. Ma on jednak kilka
wad, z których najstotniejszymi są konieczność umieszczenia na robotach
(swoich i przeciwnika) wzoru niezgodnego z regułami ligi
i słaba odporność na zmiany oświetlenia. 

W powyższym kontekście widać potrzebę stworzenia nowego systemu lokalizacji 
robotów. Techniczna możliwość stworzenia produktu spełniającego oczekiwania nie
może być kwestionowana -- w rozgrywkach ligi FIRA uczestniczy wiele drużyn, 
z których większość posiada adekwatne systemy wizyjne. Problemem może być wybór
odpowiednich metod przetwarzania obrazu oraz spełnienie kryterium
wydajnościowego.

\section{Zakres funkcjonalności}
%\section{Functional scope}
\label{sec:zakres-funkcjonalnosci}
% 
% \emph{Kontekst użytkowania produktu (aktorzy, współpracujące systemy)
%   oraz najważniejsze wymagania funkcjonalne i niefunkcjonalne.}

System ma być dostrajany przez operatora i komunikować się z programami
sterującymi robotami, które mogą znajdować się na innych 
komputerach w sieci.

Wymagania funkcjonalne:
\begin{enumerate}
\item Znajdowanie pozycji piłki i wszystkich robotów na boisku
\item Znajdowanie kąta obrotu i ientyfikacja poszczególnych robotów
we własnej drużynie
\item Zachowanie sposobu komunikacji z programami sterującymi --- poprzedni
system wizyjny komunikował się z drużynami za pomocą pakietów UDP, których
format powinien pozostać bez zmian w nowym systemie
\item Wykorzystanie kamer dostępnych w Laboratorium Robotów
\item Możliwość obserwowania sposobu i wyników działania algorytmu w czasie
rzeczywistym i wprowadzania poprawek
\item Możliwość pracy dla jednej lub dwóch drużyn --- oprócz trybu meczu, w
którym jedna z drużyn posiada nieznany kolorowy wzór na robotach system powinien
udostępniać tryb sparingu, w którym obydwie drużyny są znane i dostają 
informacje o położeniu z tego samego źródła
\item Możliwość pracy w warunkach oświetleniowych niezgodnych z regułami ligi
FIRA, w tym w zmiennych warunkach oświetleniowych
\item Możliwość użycia systemu na boisku o wymiarach niezgodnych z regułami gry
i/lub z kamerą umieszczoną na nieprzepisowej wysokości
\end{enumerate}

Wymagania niefunkcjonalne:
\begin{enumerate}
\item wydajność systemu powinna pozwalać na przetwarzanie z minimalną
częstotliwością 30 klatek na sekundę, przy możliwie najmniejszym wykorzystaniu
procesora
\end{enumerate}



\section{Wybrane aspekty realizacji}
%\section{Selected realization aspects}
\label{sec:wybrane-aspekty-realizacji}


% \emph{Przyjęte założenia, struktura i zasada działania systemu,
%   wykorzystane rozwiązania technologiczne wraz z krótkim uzasadnieniem
%   ich wyboru.}

Głównym celem projektu jest stworzenie implementacji algorytmu rozpoznawania
obrazu działającego w czasie rzeczywistym. Implementacja została wyodrębniona
w postaci biblioteki z uwagi na niezależność od pozostałych aspektów systemu --
algorytm przyjmuje obraz wejściowy i przetwarza go na zestaw współrzędnych
położenia obiektów zainteresowania. Biblioteka do lokalizacji zależy jedynie od 
OpenCV \cite{opencv}, która wykorzystywana jest do przeprowadzania 
elementarnych operacji na fragmentach obrazu. Biblioteka nie posiada żadnych 
zależności od systemu Windows, a jej API w języku C zapewnia komunikację z 
dowolnym środowiskiem programistycznym.

W konstrukcji samego algorytmu wykorzystane zostały artykuły opisujące
implementacje podobnych systemów\cite{colortag,largeleague,exemplary}. 
Niektóre elementy algorytmu, takie jak wykorzystanie
reprezentacji kolorów HSL i regresji liniowej zostały z nich zapożyczone. Inny
był natomiast projekt wzoru identyfikacyjnego umieszczanego na robotach.

Gotowa aplikacja musi być odpowiedzialna za komunikację z kamerą, 
konsumentami danych lokalizacyjnych i użytkownikiem nadzorującym pracę systemu
rozpoznawania. Napisana została jako aplikacja systemu
Windows (.NET). Program używa odziedziczonej biblioteki do pobierania ramek
obrazu z kamery. Wyniki rozpoznania są wysyłane jako pakiety UDP zgodnie z
formatem używanym przez poprzedni system lokalizacji.


% \cite{mshift}





\section{Organizacja pracy}
%\section{Work organization}
\label{sec:organizacja-pracy}
% 
% \emph{Struktura zespołu (role poszczególnych osób), krótki opis i
%   uzasadnienie przyjętej metodyki i/lub kolejności prac, planowane i
%   zrealizowane etapy prac ze wskazaniem udziału poszczególnych
%   członków zespołu, wykorzystane praktyki i narzędzia w zarządzaniu
%   projektem.}

Projekt został podzielony na część algorytmiczną (biblioteka z interfejsem w C)
i część komunikacyjną (aplikacja graficzna, odbieranie ramek z kamery i
wysyłanie wyników lokalizacji).

Z uwagi na jednokierunkowy przepływ danych (obraz $\to$ współrzędne) było 
stosunkowo prosto stworzyć środowisko testowania algorytmu off-line. Do 
podstawowej walidacji algorytmu wystarczyły zdjęcia z kamery. W ramach 
algorytmu istniały również zależności 
sekwencyjne: identyfikacja robotów w ramach drużyny wymagała wcześniejszego 
rozpoznania pozycji robotów. Interfejs graficzny pojawił się później. 

Do zarządzania kodem został wykorzystany system Mercurial
(\url{http://mercurial.selenic.com/}). 
W repozytorium znajdował się kod źródłowy, pliki obrazów testowych, programy
pomocnicze i dokumentacja w formacie Sphinx (\url{http://sphinx.pocoo.org/}).

\subsection{Szczegółowy podział prac}
Adam Chrabąszcz:
\begin{enumerate}
\item Rozpoznanie podejść, wyszukiwanie atrykułów i projektów związanych z
tematem
\item Przygotowanie prototypu algorytmu rozpoznawania pozycji robotów,
analiza wyników balansu bieli i klasyfikacji barw na dostępnych zdjęciach z
kamery
\item Biblioteka w C++ do rozpoznawania pozycji robotów. Efektywność
rozpoznania i wydajność jest weryfikowana przy pomocy wcześniej wykonanych 
zdjęć.
\item Kod identyfikujący poszczególne roboty w drużynie
\item Poprawa wydajności. Najdłużej wykonujące się fragmenty algorytmu 
przepisane do formy minimalizującej ilość operacji procesora.
\item Ustalenie ostatecznego interfejsu biblioteki w C
\end{enumerate}
Konrad Kras:
\begin{enumerate}
\item Przygotowanie projektu koszulek dla robotów, dobór odpowiedniego papieru
kolorowego do wykonania pól barwnych, wykonanie zdjęć egzemplarzy koszulek
kamerą internetową.
\item Aplikacja z graficznym interfejsem użytkownika
\item Integracja z zewnętrznymi systemami. Wczytywanie konfiguracji z pliku XML.
Wysyłanie wyników lokalizacji w postaci pakietów UDP.
\end{enumerate}

\section{Wyniki projektu}
%\section{Project results}

\label{sec:wyniki-projektu}
% 
% \emph{Najważniejsze wyniki (co konkretnie udało się uzyskać:
%   oprogramowanie, dokumentacja, raporty z testów/wdrożenia, itd.)
%   i ocena ich użyteczności (jak zostało to zweryfikowane --- np.\ wnioski
%   klienta/użytkownika, zrealizowane testy wydajnościowe, itd.),
%   istniejące ograniczenia i propozycje dalszych prac.}

Stworzenie algorytmu lokalizacji wymagało
zrozumienia wielu aspektów problemu. Szczegółowy opis algorytmu i motywacje 
kryjące się za jego konstrukcją zostały opisane w dokumentacji technicznej. 
W dokumentacji użytkownika znajduje się opis interfesu biblioteki, a także opis
aplikacji ją wykorzystującej. Dla użytkownika aplikacji przeznaczone są 
podstawowe informacje na temat parametrów, które muszą zostać dopasowane w 
programie, opis interfejsu graficznego i formatu konfiguracji XML.

W przebiegu projektu powstała biblioteka realizująca
rozpoznawanie pozycji robotów oraz wykorzystująca ją aplikacja przeprowadzająca
lokalizację w czasie rzeczywistym. 

Wydajność aplikacji zmierzono na komputerze znajdującym się w Laboratorium 
Robotów. Wynikiem było około 20
klatek na sekundę. Oznacza to o połowę większe średnie czasy pomiędzy kolejnymi
odczytami, niż zakładano. Taki wynik został uzyskany po wielokrotnych
działaniach przyspieszających implementację. Pomimo wyeliminowania największych
nieefektywności z implementacji, istnieje pole do dalszych usprawnień. Z drugiej
strony, przeprowadzanie samej lokalizacji zajmowało na innnym
komputerze ok. 9ms (110fps), więc różnice w szybkości procesora są tu bardzo 
istotne.

Efektywność rozpoznania znajduje się na dobrym poziomie. Poprawność 
rozpoznawania pozycji robotów jest praktycznie stuprocentowa. Wyznaczanie kąta 
obrotu jest mniej efektywne -- zdarzają się niepoprawne rozpoznania. Błędy nie 
występują zawsze, ale mogą pojawiać się roboty, które będąc w specyficznej 
pozycji co pewną liczbę klatek otrzymują błędny kąt obrotu. Błędy w rozpoznaniu
można redukować dokładniejszym doborem parametrów algorytmu.

Przykładowe elementy, które można poprawić:
\begin{enumerate}
\item Sposób wyznaczania kąta obrotu robota należy poprawić --- obecny
sposób wyznaczania kąta może wykrywać wartość przesuniętą o $180^\circ$.
\item Balans bieli nie bierze pod uwagę zwiększenia jasności na środku boiska.
Poprawić sytuację mógłoby zmodyfikowanie sposobu normalizacji jasności/balansu 
bieli, przez wykonanie procedury balansu przed meczem, z wykorzystaniem np.
białych kartek papieru kładzionych na boisku.
\item Wydajność algorytmu można zwiększyć, łącząc oddzielne fazy agorytmu,  
co zminimalizuje ilość dostępów do pamięci
\end{enumerate}



\bibliographystyle{plain}
\bibliography{bibliografia}


\end{document}
