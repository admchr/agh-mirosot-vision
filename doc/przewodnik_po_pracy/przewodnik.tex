% By zmienic jezyk na angielski/polski, dodaj opcje do klasy english lub polish
\documentclass[polish,12pt]{aghthesis}
\usepackage[utf8]{inputenc}
\usepackage{url}


\author{Adam Chrabąszcz, Konrad Kras}

\title{Biblioteka do lokalizacji robotów mobilnych ligi FIRA MiroSot}

\supervisor{dr inż. Wojciech Turek}

\date{2012}

% Szablon przystosowany jest do druku dwustronnego, 
\begin{document}

\maketitle



\section{Cel prac i wizja produktu}
%\section{Project goals and vision}
\label{sec:cel-wizja}
\emph{Charakterystyka problemu, motywacja projektu (w tym przegląd
  istniejących rozwiązań prowadząca do uzasadnienia celu prac), ogólna
  wizja produktu, krótkie studium wykonalności i analiza zagrożeń.}

\section{Zakres funkcjonalności}
%\section{Functional scope}
\label{sec:zakres-funkcjonalnosci}

\emph{Kontekst użytkowania produktu (aktorzy, współpracujące systemy)
  oraz najważniejsze wymagania funkcjonalne i niefunkcjonalne.}

\section{Wybrane aspekty realizacji}
%\section{Selected realization aspects}
\label{sec:wybrane-aspekty-realizacji}

\cite{opencv}
\cite{colortag}
\cite{exemplary}
\cite{largeleague}
\cite{mshift}

\emph{Przyjęte założenia, struktura i zasada działania systemu,
  wykorzystane rozwiązania technologiczne wraz z krótkim uzasadnieniem
  ich wyboru.}

\section{Organizacja pracy}
%\section{Work organization}
\label{sec:organizacja-pracy}

\emph{Struktura zespołu (role poszczególnych osób), krótki opis i
  uzasadnienie przyjętej metodyki i/lub kolejności prac, planowane i
  zrealizowane etapy prac ze wskazaniem udziału poszczególnych
  członków zespołu, wykorzystane praktyki i narzędzia w zarządzaniu
  projektem.}

\section{Wyniki projektu}
%\section{Project results}

\label{sec:wyniki-projektu}

\emph{Najważniejsze wyniki (co konkretnie udało się uzyskać:
  oprogramowanie, dokumentacja, raporty z testów/wdrożenia, itd.)
  i ocena ich użyteczności (jak zostało to zweryfikowane --- np.\ wnioski
  klienta/użytkownika, zrealizowane testy wydajnościowe, itd.),
  istniejące ograniczenia i propozycje dalszych prac.}

% o ile to mozliwe prosze uzywac odwolan \cite w konkretnych miejscach a nie \nocite

\nocite{artykul2011,ksiazka2011,narzedzie2011,projekt2011}

\bibliography{bibliografia}

\end{document}
