% By zmienic jezyk na angielski/polski, dodaj opcje do klasy english lub polish
\documentclass[polish,12pt]{aghthesis}
\usepackage[latin2]{inputenc}
\usepackage{url}

\author{Jan Kowalski, Jan Malinowski\\ Wojciech Kowalski}

\title{Zarys szczeg�lnych podstaw\\ og�lnej teorii wszystkiego}

\supervisor{dr hab.\ in�.\ Krzysztof Iksi�ski, prof.\ nadzw.\ AGH}

\date{2011}

% Szablon przystosowany jest do druku dwustronnego, 
\begin{document}

\maketitle



\section{Cel prac i wizja produktu}
%\section{Project goals and vision}
\label{sec:cel-wizja}
\emph{Charakterystyka problemu, motywacja projektu (w tym przegl�d
  istniej�cych rozwi�za� prowadz�ca do uzasadnienia celu prac), og�lna
  wizja produktu, kr�tkie studium wykonalno�ci i analiza zagro�e�.}

\section{Zakres funkcjonalno�ci}
%\section{Functional scope}
\label{sec:zakres-funkcjonalnosci}

\emph{Kontekst u�ytkowania produktu (aktorzy, wsp�pracuj�ce systemy)
  oraz najwa�niejsze wymagania funkcjonalne i niefunkcjonalne.}

\section{Wybrane aspekty realizacji}
%\section{Selected realization aspects}
\label{sec:wybrane-aspekty-realizacji}

\emph{Przyj�te za�o�enia, struktura i zasada dzia�ania systemu,
  wykorzystane rozwi�zania technologiczne wraz z kr�tkim uzasadnieniem
  ich wyboru.}

\section{Organizacja pracy}
%\section{Work organization}
\label{sec:organizacja-pracy}

\emph{Struktura zespo�u (role poszczeg�lnych os�b), kr�tki opis i
  uzasadnienie przyj�tej metodyki i/lub kolejno�ci prac, planowane i
  zrealizowane etapy prac ze wskazaniem udzia�u poszczeg�lnych
  cz�onk�w zespo�u, wykorzystane praktyki i narz�dzia w zarz�dzaniu
  projektem.}

\section{Wyniki projektu}
%\section{Project results}

\label{sec:wyniki-projektu}

\emph{Najwa�niejsze wyniki (co konkretnie uda�o si� uzyska�:
  oprogramowanie, dokumentacja, raporty z test�w/wdro�enia, itd.)
  i ocena ich u�yteczno�ci (jak zosta�o to zweryfikowane --- np.\ wnioski
  klienta/u�ytkownika, zrealizowane testy wydajno�ciowe, itd.),
  istniej�ce ograniczenia i propozycje dalszych prac.}

% o ile to mozliwe prosze uzywac odwolan \cite w konkretnych miejscach a nie \nocite

\nocite{artykul2011,ksiazka2011,narzedzie2011,projekt2011}

\bibliography{bibliografia}

\end{document}
