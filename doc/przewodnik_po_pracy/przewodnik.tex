% By zmienic jezyk na angielski/polski, dodaj opcje do klasy english lub polish
\documentclass[polish,12pt]{aghthesis}
\usepackage[utf8]{inputenc}
\usepackage{url}


\author{Adam Chrabąszcz, Konrad Kras}

\title{Biblioteka do lokalizacji robotów mobilnych ligi FIRA MiroSot}

\supervisor{dr inż. Wojciech Turek}

\date{2012}

% Szablon przystosowany jest do druku dwustronnego, 
\begin{document}

\maketitle



\section{Cel prac i wizja produktu}
%\section{Project goals and vision}
\label{sec:cel-wizja}
% \emph{Charakterystyka problemu, motywacja projektu (w tym przegląd
%   istniejących rozwiązań prowadząca do uzasadnienia celu prac), ogólna
%   wizja produktu, krótkie studium wykonalności i analiza zagrożeń.}
Celem projektu jest
dostarczenie systemu umożliwiającego rozpoznawanie w czasie rzeczywistym
pozycji robotów oraz piłki na obrazie z kamery nad boiskiem gry zgodnie z
regułami ligi FIRA MiroSot Small League.

Liga FIRA MiroSot Small League określa zbiór zasad gry, w której drużyny
składające się z pięciu 7.5cm robotów poruszających się po prostokątnym boisku
220cm na 180cm próbują umieścić piłkę golfową w bramce przeciwnika. Roboty danej
drużyny sterowane są za pomocą jednego komputera, który uzyskuje informacje
o sytuacji na boisku przy pomocy kamery znajdującej się centralnie nad boiskiem.
W Laboratorium Robotów Katedry Informatyki znajdują się wszelkie fizyczne zasoby
niezbędne do rozegrania meczu zgodnie z zasadami ligi: zestaw robotów, 
odpowiednie boisko oraz kamera. Rozgrywanie meczy uniemożliwia brak 
odpowiedniego oprogramowania sterującego robotami. W skład takiego 
oprogramowania musi wchodzić system rozpoznający obraz z kamery. 

Aktualnie w laboratorium działa system rozpoznawania robotów. Ma on jednak kilka
wad, z których najstotniejszymi są konieczność umieszczenia na robotach
(swoich i przeciwnika) wzoru niezgodnego z regułami ligi
i słaba odporność na zmiany oświetlenia.
W trakcie poszukiwania materiałów nie znaleźliśmy projektów udostępniających
kod gotowych rozwiązań problemu. Jedyny projekt, który znaleźliśmy dotyczył
innego zestawu reguł gry i nie mógł zostać zastosowany. 

Możliwość stworzenia produktu spełniającego oczekiwania nie może być
kwestionowana -- w rozgrywkach ligi FIRA uczestniczy wiele drużyn, z których
większość posiada adekwatne systemy wizyjne. Problemem może być wybór
odpowiednich metod przetwarzania obrazu oraz spełnienie kryterium
wydajnościowego.

\section{Zakres funkcjonalności}
%\section{Functional scope}
\label{sec:zakres-funkcjonalnosci}
% 
% \emph{Kontekst użytkowania produktu (aktorzy, współpracujące systemy)
%   oraz najważniejsze wymagania funkcjonalne i niefunkcjonalne.}

System ma być dostrajany przez operatora i komunikować się z programami
sterującymi robotami (które mogą znajdować się na innych 
komputerach w sieci).

Wymagania funkcjonalne:
\begin{enumerate}
\item zachowanie sposobu komunikacji z programami sterującymi --- poprzedni
system wizyjny komunikował się z drużynami za pomocą pakietów UDP, których
format powinien pozostać bez zmian w nowym systemie
\item współpraca z kamerami, które były używane w poprzednim systemie
\item możliwość obserwowania sposobu i wyników działania algorytmu w czasie
rzeczywistym i wprowadzania poprawek
\item możliwość pracy dla jednej lub dwóch drużyn --- oprócz trybu meczu, w 
którym jedna z drużyn posiada nieznany kolorowy wzór na robotach system powinien
udostępniać tryb sparingu, w którym obydwie drużyny są znane i dostają 
informacje o położeniu z tego samego źródła
\item możliwość pracy w warunkach oświetleniowych niezgodnych z regułami ligi 
FIRA, w tym w zmiennych warunkach oświetleniowych
\item możliwość użycia systemu na boisku o wymiarach niezgodnych z regułami gry
i/lub z kamerą umieszczoną na nieprzepisowej wysokości

\end{enumerate}
Wymagania niefunkcjonalne:
\begin{enumerate}
\item wydajność systemu powinna pozwalać na przetwarzanie z minimalną
częstotliwością 30 klatek na sekundę, przy możliwie najmniejszym wykorzystaniu
procesora
\end{enumerate}



\section{Wybrane aspekty realizacji}
%\section{Selected realization aspects}
\label{sec:wybrane-aspekty-realizacji}


% \emph{Przyjęte założenia, struktura i zasada działania systemu,
%   wykorzystane rozwiązania technologiczne wraz z krótkim uzasadnieniem
%   ich wyboru.}

Głównym celem projektu jest stworzenie implementacji algorytmu rozpoznawania
obrazu działającego w czasie rzeczywistym. Implementacja została wyodrębniona
w postaci biblioteki z uwagi na niezależność od pozostałych aspektów systemu --
algorytm przyjmuje obraz wejściowy i przetwarza go na zestaw współrzędnych
położenia obiektów zainteresowania. Biblioteka do lokalizacji zależy jedynie od 
OpenCV \cite{opencv}, która wykorzystywana jest do przeprowadzania 
elementarnych operacji na fragmentach obrazu. Biblioteka nie posiada żadnych 
zależności od systemu Windows, a jej API w języku C zapewnia komunikację z 
dowolnym środowiskiem programistycznym.

W konstrukcji samego algorytmu wykorzystane zostały artykuły opisujące 
implementacje podobnych systemów w \cite{largeleague} \cite{exemplary}
\cite{colortag}. Główne elementy konstrukcji algorytmu, takie jak wykorzystanie
reprezentacji kolorów HSL i regresji Deminga zostały z nich zapożyczone. Własny
był natomiast projekt wzoru identyfikacyjnego umieszczanego na robotach, który
posiadał większe pola przeznaczone na kolory identyfikacyjne.

Fragment systemu odpowiedzialny za komunikację z kamerą, użytkownikiem 
nadzorującym precę systemu i konsumentami danych 
wyjściowych algorytmu rozpoznawania napisany został jako aplikacja systemu
Windows (.NET). Program używa odziedziczonej biblioteki do pobierania ramek
obrazu z kamery. Wyniki rozpoznania są wysyłane jako pakiety UDP zgodnie z
formatem używanym przez poprzedni system rozpoznawania.


% \cite{mshift}





\section{Organizacja pracy}
%\section{Work organization}
\label{sec:organizacja-pracy}
% 
% \emph{Struktura zespołu (role poszczególnych osób), krótki opis i
%   uzasadnienie przyjętej metodyki i/lub kolejności prac, planowane i
%   zrealizowane etapy prac ze wskazaniem udziału poszczególnych
%   członków zespołu, wykorzystane praktyki i narzędzia w zarządzaniu
%   projektem.}

\begin{enumerate}
\item Rozpoznanie podejść, wyszukiwanie atrykułów i projektów związanych z
tematem. (Adam Chrabąszcz)
\item Przygotowanie prototypu algorytmu rozpoznawania pozycji robotów,
analiza wyników balansu bieli i klasyfikacji barw na dostępnych zdjęciach z 
kamery. (Adam Chrabąszcz)
\item Przygotowanie projektu koszulek dla robotów, dobór odpowiedniego papieru
kolorowego do wykonania pól barwnych, wykonanie zdjęć egzemplarzy koszulek 
kamerą internetową. (Konrad Kras)
\item Walidacja podejścia za pomocą wcześniej wykonanych zdjęć.
\end{enumerate}


\section{Wyniki projektu}
%\section{Project results}

\label{sec:wyniki-projektu}
% 
% \emph{Najważniejsze wyniki (co konkretnie udało się uzyskać:
%   oprogramowanie, dokumentacja, raporty z testów/wdrożenia, itd.)
%   i ocena ich użyteczności (jak zostało to zweryfikowane --- np.\ wnioski
%   klienta/użytkownika, zrealizowane testy wydajnościowe, itd.),
%   istniejące ograniczenia i propozycje dalszych prac.}

\bibliography{bibliografia}

\end{document}
